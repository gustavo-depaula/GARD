%*******************************************************
% Abstract
%*******************************************************
%\renewcommand{\abstractname}{Abstract}
\pdfbookmark[1]{Abstract}{Abstract}
% \addcontentsline{toc}{chapter}{\tocEntry{Abstract}}
\begingroup
% \let\clearpage\relax
% \let\cleardoublepage\relax
% \let\cleardoublepage\relax

\chapter*{Abstract}

Vision-based landing systems have gained increased attention as cost-effective and autonomous alternatives 
to traditional radio-based autoland systems, particularly in scenarios where on-ground infrastructure is 
unavailable or infeasible. A major bottleneck in advancing this technology is the lack of large-scale, diverse, 
and realistic runway image datasets with pixel-level annotations. While recent works have turned to synthetic 
data generated from flight simulators, these approaches are limited by their lack of scalability, poor 
environmental variability, and the high cost of manual annotation.

This paper introduces a novel, open-source, modular data augmentation pipeline, called \emph{Canny2Concrete}, 
that leverages latent diffusion models with ControlNet to generate high-resolution, labeled runway images. 
By conditioning image generation on structural features extracted from existing datasets, the pipeline produces 
realistic synthetic images that preserve critical runway geometry while offering substantial diversity in terms 
of weather, lighting, and background scenery. A three-stage variant generation process enhances image variety 
through positional transformations, outpainting, and environmental occlusion effects.

Using this pipeline, a large-scale synthetic runway dataset---\emph{GARD: Gustavo's Awesome Runway Dataset}---has 
been constructed and evaluated both intrinsically, using the Structural Similarity Index (SSIM), and extrinsically, 
by training and fine-tuning state-of-the-art detection and segmentation models. Results demonstrate that models 
trained on the proposed dataset match or outperform those trained on existing synthetic datasets, confirming 
the viability of diffusion-based augmentation for runway segmentation tasks, and the effectiveness of the
\emph{Canny2Concrete} pipeline in generating realistic runway images. The complete dataset, pipeline, and 
evaluation tools are publicly available to support further research in autonomous aviation and computer vision.

\emph{GARD}, along with trained segmentation model weights and evaluation results, is available at:
\begin{center}
\url{https://www.kaggle.com/datasets/depaulagu/gard2025}
\end{center}

The code for the project, including the implementation of the \emph{Canny2Concrete} pipeline, is available at:
\begin{center}
\url{https://github.com/gustavo-depaula/GARD}
\end{center}

% \vfill

\endgroup

