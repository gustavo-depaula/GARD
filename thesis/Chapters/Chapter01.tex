%************************************************
\chapter{Introduction}\label{ch:introduction}
%************************************************
Aviation is regarded as highly safe medium of transportation, and is marked by high-degrees of automation, reducing the biggest cause (85\% of the general aviation crashes) of accidents: pilot error \cite{li_factors_2001}. The increase in automation reduces cognitive, fatigue, and inexperience risks for pilots and is and important factor, alongside better training and safety regulations, to the to the rapidly decrease (1959-2024) in the fatal accidents rate \cite{airbus_fatal_nodate}.

Although the approach and landing phases are the minority of the flight time, they contribute disproportionally to the number of accidents, this data being corroborated by the two major commercial airplane companies Boeing (7\% in approach, 36\% in landing) \cite{boeing_statistical_2024}, and Airbus (59\% in landing, 11\% in approach) \cite{airbus_accidents_nodate}.

Contributing as a risk-factor, the approach and landing phases are flight parts that require heavy human intervention. This has led to an increased interest in the development of autonomous landing (autoland) systems, which are systems that can autonomously navigate the civil aircraft or UAV (unmanned aerial vehicle, e.g. drones) during the landing procedure. Currently, most autoland systems are based on radio signals that provide guidance to the system, such as ILS (instrument landing system) and PAR (precision approach radar).

Radio-based autonomous landing systems have the advantage of allowing the landing in extremely adverse weather conditions and low visibility, but they have a high-cost of deployment and maintenance, can suffer from electromagnetic and radio interference, and require on-the-ground specialized equipment to support the aircraft (e.g. localizer and glideslope).

The recent advancements in the Computer Vision field have sparked increasing interest \cite{airbus_airbus_2021} in developing vision-based autoland systems, which use visual navigation to guide the aircraft during the approach and landing phases. In \cite{xin_vision-based_2022}, the authors describe key advantages of vision-based autoland systems for UAV: autonomy, low cost, resistance to interference, and ability to be combined with other navigation methods for higher accuracy. Vision based landing is especially attractive for drones that often need to land in extreme situations when in military use, environmental monitoring, and disaster relief, where runways may not have the necessary equipment for radio-based systems.

Two key parts of a autonomous landing system is detection and segmentation of runways, the former being responsible for detecting the presence and locating the runway on the image by drawing a bounding box around the runway, and the latter, more granularly, works at the pixel-level, identifying the exact shape of the object on the image.

Detection and segmentation methods can be separated into two broad fields: traditional methods and deep-learning based methods. Traditional methods employ handcrafted features and mathematical rules to segment images, e.g. recognition of texture, line segments, and shape features of runways \cite{aytekin_texture-based_2013} \cite{ye_research_2020}. They tend to be faster to run and not require training, but because they require handcrafted rules, they often fail to generalize to out-of-sample images and complex real-world scenarios that involve similar objects like roads and multiple runways and adverse weather conditions.

Compared to traditional methods, deep-learning based ones have shown improved generalization and performance on out-of-sample images, with the latest published runway segmentation papers all having used this approach and getting better results than traditional methods \cite{chen_image-based_2024} \cite{wang_valnet_2024}.

However, because of their data-hungry nature, previous research \cite{wang_valnet_2024} \cite{chen_image-based_2024} \cite{chen_bars_2023} \cite{ducoffe_lard_2023} all note the lack of publicly available, high-quality, large, real-world datasets of aeronautical images to be used in the task of runway segmentation. To close this gap, the researchers have built and published datasets comprised of synthetic images collected from flight simulators such as X-Plane \cite{chen_bars_2023} \cite{wang_valnet_2024}, Microsoft Flight Simulator \cite{chen_image-based_2024}, and images collected from Google Earth Studio \cite{ducoffe_lard_2023}.

While these synthetic datasets have contributed to advancements in the field, there still an opportunity of an open-source, realistic, high-quality dataset that has runway image data encompassing a greater variety of weather and lighting conditions, environments and runway structures, and, even better, an open-source image generator that is capable of generating novel images or augmenting an existing dataset.

In recent years, striking advancements were made in the field of image generation models, with several closed-source and open-source models being published for general use, such as DALL-E \cite{betker_improving_nodate}, Midjourney \cite{midjourney_midjourney_nodate}, Kandinsky \cite{arkhipkin_kandinsky_2024}, and Stable Diffusion \cite{rombach_high-resolution_2022}. These recently state-of-the-art models are latent diffusion models (LDM), which have surpassed in performance the previous state-of-the-art techniques that were based in generative adversarial networks (GANs). These image-generation models are now so capable of generating realistic images that researchers are studying the impacts of generative AI on the spread of fake news \cite{loth_blessing_2024}.

These recent research advancements opened a new opportunity window to use synthetic data to close the gap on the lack of real images in object detection and segmentation tasks. There have already been progress in this idea applied to other areas such as medical images \cite{saragih_using_2024}, urban applications \cite{reutov_generating_2023}, and apple detection in orchards \cite{voetman_big_2023}. But up to my knowledge of public research, there has been no generative AI-based open-source dataset that could be used for the task of runway detection and segmentation.

To address the challenges of lack of data in the context of runway segmentation tasks, this paper introduces a novel, open-source, data augmentation technique based on a multi-step Stable Diffusion pipeline that extracts features existing datasets and outputs images that retain a similar structure but are customizable in respect to scenery, weather, and lighting conditions, guided by a text prompt. The images generated by the pipeline are already labeled and don't require handcraft labelling.

This technique is then used to construct a novel, large-scale, high-resolution, open-source, dataset of labeled runway images with good coverage of image variants in respect to weather, lighting, background, and runway occlusion.

This approach can greatly increase the availability of runway images to be used in other research projects, and, by virtue of being open-source, allow other researchers to generate and augment their own synthetic datasets with their own desired characteristics.

This new dataset is evaluated both theoretically by measuring the similarity of generated images to real images using metrics such as Fréchet Inception Distance (FID) and Structural Similarity Index (SSIM), and practically by introducing a runway detection and segmentation pipeline based on state-of-the-art models and comparing the performance of the model when training it with an existing dataset and when training with the new dataset.
