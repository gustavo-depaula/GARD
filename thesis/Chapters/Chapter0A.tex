%********************************************************************
% Appendix
%*******************************************************
% If problems with the headers: get headings in appendix etc. right
%\markboth{\spacedlowsmallcaps{Appendix}}{\spacedlowsmallcaps{Appendix}}
\chapter{Appendix}


\section{Hardware and Software Specifications}

All the experiments were conducted on a computer with the following
specifications:
\graffito{I use arch btw.}
\begin{itemize}
    \item \textbf{CPU:} AMD Ryzen 9 7900X (24) @ 5.733GHz
    \item \textbf{GPU:} NVIDIA GeForce RTX 4070 Ti, 12GB
    \item \textbf{Memory:} 64 GB
    \item \textbf{Operating System:} Arch Linux x86\_64
    \item \textbf{Linux Kernel:} 6.13.6-arch1-1
\end{itemize}

\section{Experiments duration}

To generate the base images, about 10 hours were needed. To generate
the variant images, about 40 hours were needed. To generate the variant images
with occlusion, about 2 hours were needed.

\section{JSON label properties}

In the Table~\ref{tab:json_fields_summary}, we summarize the most important top-level fields in the JSON label file describing runway image metadata, annotations, generation, and transformation history.

\begin{table}[htbp]

\centering
\small
\renewcommand{\arraystretch}{1.3}
\begin{tabular}{|p{4cm}|p{9.5cm}|}
\hline
\textbf{Property} & \textbf{Description} \\
\hline
\texttt{dataset} & Name of the dataset the image belongs to (e.g., \texttt{LARD}). Useful for classification or filtering. \\
\hline
\texttt{sourceImage} & Filename of the original template image. \\
\hline
\texttt{runwayLabel} & List of 4 \texttt{[x, y]} coordinates representing the annotated runway corners (pre-transform). \\
\hline
\texttt{variant} & Visual variant of the scene (e.g., \texttt{dawn}, \texttt{night}, etc.). Useful for filtering by scene conditions. \\
\hline
\texttt{prompt} & Positive text prompt used for image generation (e.g., in Stable Diffusion). Critical for reproducibility. \\
\hline
\texttt{negative\_prompt} & Negative prompt describing what the model should avoid during image generation. \\
\hline
\texttt{seed} & Random seed used for initial image generation — ensures image reproducibility. \\
\hline
\texttt{model} & Name of the diffusion model used (e.g., \texttt{sdxl-dreamshaperxl}). \\
\hline
\texttt{baseImage} & Identifier for the untransformed version of the generated image, before augmentations. \\
\hline
\texttt{albumentationsReplay} & Complete record of image transformations (flips, crops, affine) used for augmentation — including parameters and outcomes. Enables full replay of augmentations. \\
\hline
\texttt{outpaintingSeed} & Seed used during outpainting generation (if enabled). \\
\hline
\texttt{occlusionEffects} & Details of occlusion layers applied (e.g., clouds), including seed used for randomness. \\
\hline
\end{tabular}
\caption{Top-level fields in the JSON label file describing runway image metadata, annotations, generation, and transformation history.}
\label{tab:json_fields_summary}
\end{table}

